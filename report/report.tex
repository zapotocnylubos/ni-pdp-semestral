\documentclass[epsf,epic,eepic,eepicemu]{article}\oddsidemargin=-5mm
\evensidemargin=-5mm\marginparwidth=.08in \marginparsep=.01in
\marginparpush=5pt\topmargin=-15mm\headheight=12pt
\textwidth=15cm\columnsep=2mm
\columnseprule=1pt\parindent=15pt\parskip=4pt

\usepackage[T1]{fontenc}
\usepackage[czech]{babel} % recommended if you write in Czech

\usepackage{xcolor}
\definecolor{codeblue}{rgb}{0.2,0.2,1}
\definecolor{codered}{rgb}{1,0.2,0.2}
\definecolor{codegreen}{rgb}{0.2,0.6,0.2}
\definecolor{codemagenta}{rgb}{0.6,0.2,0.6}

\usepackage{float}

\usepackage{listings}
\lstset{
    basicstyle=\ttfamily\small,
    language=C,
    tabsize=2,
    showstringspaces=false,
    breaklines=true,
    captionpos=b,
    keywordstyle=\color{codeblue}\ttfamily,
    stringstyle=\color{codered}\ttfamily,
    commentstyle=\color{codegreen}\ttfamily,
    morecomment=[l][\color{codemagenta}]{\#}
}

\usepackage{caption}
\renewcommand{\lstlistingname}{Kód}% Listing -> Kód

\begin{document}
\begin{center}
\bf Semestrální projekt NI-PDP 2022/2023\\[5mm]
    Paralelní algoritmus pro hledání\\minimálního hranového řezu hranově ohodnoceného grafu\\[5mm] 
       Bc. Luboš Zápotočný\\[5mm]
FIT ČVUT\\Thákurova 9, 160 00 Praha 6\\[2mm]
\today
\end{center}

\section{Definice pojmů a popis sekvenčního algoritmu}

V této kapitole se budeme věnovat definici problému a popisu sekvenčního algoritmu pro řešení minimálního hranového řezu v hranově ohodnoceném grafu. Nejprve připomeneme základní pojmy týkající se grafů a hranového řezu a poté představíme konkrétní problém, který se v této práci řešil.

Problém minimálního hranového řezu v hranově ohodnoceném grafu spočívá v nalezení dvou disjunktních podmnožin uzlů, takzvaných X a Y, takových, že součet ohodnocení všech hran spojujících uzly z obou množin je minimální. Tento problém má mnoho praktických aplikací, například v oblasti sítí, kde se hledají nejlevnější cesty mezi uzly sítě.

Vstupní data obsahují číslo reprezentující počet vrcholů v grafu a následně reprezentaci ohodnocení hran pomocí matice sousednosti. Ukázka vstupních dat je vyobrazena v ukázce \ref{lst:inputExample}.

\begin{lstlisting}[label=lst:inputExample, caption={Vstupní data},language={}]
10
     0   112     0     0    98    80     0     0    91   102
   112     0    90     0     0     0     0   119    96     0
     0    90     0     0   104   111    82     0     0   107
     0     0     0     0     0   114    96     0     0     0
    98     0   104     0     0   118    80    88     0     0
    80     0   111   114   118     0   105   106     0   105
     0     0    82    96    80   105     0   109    93    99
     0   119     0     0    88   106   109     0     0    83
    91    96     0     0     0     0    93     0     0    95
   102     0   107     0     0   105    99    83    95     0
\end{lstlisting}

Sekvenční algoritmus pro řešení tohoto problému je založen na postupném procházení některých (nikoli všech) kombinací podmnožin uzlů X a Y a hledání takové kombinace, která minimalizuje součet ohodnocení hran mezi nimi. 

Některé kombinace algoritmus vynechává, jelikož nemůžou vést k lepšímu řešení, než je aktuálně nalezené minimum. Při průchodu stromem možných kombinací některé podstromy můžeme vynechat. Této metodě se říká Branch and Bound.

Princip prořezávání pomocí Branch and Bound spočívá v systematickém prohledávání stromu kombinací s cílem najít optimální řešení s co nejmenším počtem vyhodnocených kombinací. Algoritmus začíná v kořeni stromu a postupně prochází všechny jeho větve, přičemž se snaží minimalizovat horní hranici ceny řešení a zároveň maximalizovat dolní hranici ceny řešení. Toho se dosahuje pomocí prořezávání podstromů, které nemohou obsahovat optimální řešení.

Prořezávání probíhá pomocí dvou technik: dolních a horních odhadů. Dolní odhady se používají k identifikaci podstromů, které neobsahují optimální řešení, a mohou být tedy bezpečně prořezány. Horní odhady se používají k minimalizaci počtu vyhodnocených kombinací tím, že prořezávají podstromy, jejichž řešení by bylo horší než nejlepší nalezené řešení. Ukázka kódu \ref{lst:cutUpperBound} zobrazuje implementaci horního řezu. Ukázka kódu \ref{lst:cutLowerBound} znázorněna podmínka pro řez pomocí dolního odhadu.

\begin{lstlisting}[float,label=lst:cutUpperBound, caption={Horní řez}]
if (currentWeight >= bestWeight) {
    upperBoundCounter++;
    return;
}
\end{lstlisting}

\begin{lstlisting}[float,label=lst:cutLowerBound, caption={Dolní řez}]
int lowerBound = graph.cutLowerBound(cut, index);

if (currentWeight + lowerBound >= bestWeight) {
    lowerBoundCounter++;
    return;
}
\end{lstlisting}

Dolní odhad je spočítán tak, že pro každý zatím nepřiřazený uzel grafu v daném mezistavu s částečným řezem vypočteme, o kolik by se váha řezu zvýšila, pokud by tento uzel patřil do X, o kolik by se zvýšila, pokud by tento uzel patřil do Y a vezmeme menší z těchto dvou hodnot a tato minima posčítáme pro všechny nepřiřazené uzly.

Cílem prořezávání stromu kombinací pomocí Branch and Bound je minimalizovat počet vyhodnocených kombinací a najít optimální řešení. Tento algoritmus je velmi účinný pro řešení problémů, kde je prostor kombinací velký a výpočetní čas je omezený. \cite{coursesPDPBFS}

Rekurzivní části algoritmu jsou vyobrazeny na ukázkách \ref{lst:recursiveDef} a \ref{lst:recursiveCall}.

\begin{lstlisting}[float,label=lst:recursiveDef, caption={Prototyp rekurzivní funkce}]
void DFS_BB(const Graph &graph, int maxPartitionSize,
            bool *cut, int count, int index,
            int currentWeight, int &bestWeight);
\end{lstlisting}

\begin{lstlisting}[float,label=lst:recursiveCall, caption={Rekurzivní volání}]
// try with this vertex
cut[index] = true;

// try with this vertex (need to extend current cut)
DFS_BB(graph, maxPartitionSize,
       cut, count + 1, index + 1,
       currentWeight + graph.vertexWeight(cut, index, index), bestWeight);

// restore the status of the cut
cut[index] = false;

// try without this vertex (need to extend current cut)
DFS_BB(graph, maxPartitionSize,
       cut, count, index + 1,
       currentWeight + graph.vertexWeight(cut, index, index), bestWeight);
\end{lstlisting}

Za předpokladu, že je nalezeno lepší řešení, je tato hodnota uložena a rekurzivní prohledávání této cesty ve stromu kombinací je u konce. Kód \ref{lst:recursiveBest} zobrazuje část rekurzivní funkce, která kontroluje aktuální váhu řezu a aktualizuje globální proměnnou v případě nalezení lepšího řešení.

\begin{lstlisting}[float,label=lst:recursiveBest, caption={Prototyp rekurzivní funkce}]
if (index == graph.size) {
    if (count != maxPartitionSize) {
        return;
    }

    if (currentWeight < bestWeight) {
        bestWeight = currentWeight;
    }
    
    return;
}
\end{lstlisting}



Popiste paralelni algoritmus, opet vyjdete ze zadani a presne vymezte
odchylky, ktere pri implementaci OpenMP pouzivate.
Popiste a vysvetlete strukturu celkoveho
paralelniho algoritmu na urovni procesuu v OpenMP a strukturu kodu
jednotlivych procesu. Napr. jak je naimplemtovana smycka pro cinnost
procesu v aktivnim stavu i v stavu necinnosti. Jake jste zvolili
konstanty a parametry pro skalovani algoritmu. Struktura a semantika
prikazove radky pro spousteni programu.


\section{Popis paralelniho algoritmu a jeho implementace v OpenMP - datovy paralelismus}

Popiste paralelni algoritmus, opet vyjdete ze zadani a presne vymezte
odchylky, ktere pri implementaci OpenMP pouzivate.
Popiste a vysvetlete strukturu celkoveho
paralelniho algoritmu na urovni procesuu v OpenMP a strukturu kodu
jednotlivych procesu. Napr. jak je naimplemtovana smycka pro cinnost
procesu v aktivnim stavu i v stavu necinnosti. Jake jste zvolili
konstanty a parametry pro skalovani algoritmu. Struktura a semantika
prikazove radky pro spousteni programu.

\section{Popis paralelniho algoritmu a jeho implementace v MPI}

Popiste paralelni algoritmus, opet vyjdete ze zadani a presne vymezte
odchylky, zvlaste u Master-Slave casti. Popiste a vysvetlete strukturu celkoveho
paralelniho algoritmu na urovni procesuu v MPI a strukturu kodu
jednotlivych procesu. Napr. jak je naimplemtovana smycka pro cinnost
procesu v aktivnim stavu i v stavu necinnosti. Jake jste zvolili
konstanty a parametry pro skalovani algoritmu. Struktura a semantika
prikazove radky pro spousteni programu.

\section{Namerene vysledky a vyhodnoceni}

\begin{enumerate}
\item Zvolte tri instance problemu s takovou velikosti vstupnich dat, pro ktere ma sekvencni 
algoritmus casovou slozitost alespon nekolik minut - vice informaci na {\tt http://courses.fit.cvut.cz} v sekci "Organizace cviceni".
Pro mereni cas potrebny na cteni dat z disku a ulozeni na disk neuvazujte a zakomentujte
ladici tisky, logy, zpravy a vystupy.
\item Merte paralelni cas pri pouziti $i=2,\cdot,60$ vypocetnich jader.
\item Tabulkova a pripadne graficky zpracovane namerene hodnoty casove slozitosti měernych instanci behu programu s popisem instanci dat. Z namerenych dat sestavte grafy zrychleni $S(n,p)$. 
\item Analyza a hodnoceni vlastnosti paralelniho programu, zvlaste jeho efektivnosti a skalovatelnosti, pripadne popis zjisteneho superlinearniho zrychleni.

\end{enumerate}

\section{Zaver}

Celkove zhodnoceni semestralni prace a zkusenosti ziskanych behem semestru.

\section{Literatura}

\bibliographystyle{unsrt}
\bibliography{report}

\end{document}
